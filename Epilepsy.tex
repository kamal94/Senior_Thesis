\section{Epilepsy}
\subsection{Definition}
The medical field defines a seizure as a sudden burst of electricity in the brain that causes a disturbance in its electrical activity. An epileptic seizure is more strictly defined as a "transient occurrence of signs and/or symptoms due to abnormal excessive or synchronous neuronal activity in the brain." Until recently, epilepsy has been defined as a brain disorder where the patient suffers from two unprovoked epileptic seizures in a span greater than 24 hours. In 2014, the definition of epilepsy has been revised to encompasses patients who display any of the following symptoms:

\blockquote{
\begin{itemize}
	\item At least two unprovoked (or reflex) seizures occurring in a span greater than 24 hours.
	\item One unprovoked (or reflex) seizure and a probability of further seizures similar to the general recurrence risk  (at least 60\%) after two unprovoked seizures, occurring over the next 10 years.
	\item Diagnosis of an epilepsy syndrome. Epilepsy is considered to be resolved for individuals who either had an age- dependent epilepsy syndrome but are now past the applicable age or who have remained seizure-free for the last 10 years and off antiseizure medicines for at least the last 5 years.
\end{itemize}
}

The new definition was designed to allow physicians more freedom in diagnosing cases of epilepsy and take drastic measures in special circumstances where the first occurrence of an epileptic seizure can be greatly suggestive of a problem, necessitating an urgent procedure. It also allows patients to 'outgrow' epilepsy in cases where the condition is age dependent or has not manifested for a prolonged period of time, indicating the condition no longer affects the patient. Most importantly, the new definition classifies epilepsy as a \textit{disease} rather than a disorder, an action that hopes to spread awareness regarding the "lasting derangement of normal function" that epilepsy can entail on a patient \cite{practicalDefEpi2014}.

There are two main types of epileptic seizures: general seizures and partial seizures. The distinction between the two is where they originate and how greatly they spread in the brain. In general seizures, an electric discharge spreads to both hemispheres of the brain causing the electrical dysfunction that results in the seizure. Similarly, partial (of Focal) seizures are caused by an electric discharge, but this charge spreads only to certain areas of the brain in which it causes a dysfunction leading to the seizure syndrome cite{}. 

\subsection{Syndrome}
Epilepsy patients can experience a wide range of physical syndromes, not all of which are observable to a bystander. Some observable syndromes are 
\begin{itemize}
	\item drooling; inability to swallow; difficulty talking.
	\item repeated non-purposeful movements; rigid or tense muscles; tremors, twitching or jerking.
	\item pale or flushed skin color; seating; biting of tongue.
\end{itemize}

Some non-physical signs that the patients report feeling are
\begin{itemize}
	\item loss of awareness; forgetfulness; distraction; daydreaming
	\item loss or obscurity of vision, hearing, and sensation
	\item body feels different, or out of body feeling
\end{itemize}

\subsection{Epilepsy in Numbers and Life}
In the United States alone there are 150,000 new cases of epilepsy a year, or about 48 cases for every 100,000 people. In total, there are about 2.2 million cases of epilepsy in the United States, averaging close to 7 cases for every 1000 people. The disease onset is most common in young children and older adults \cite{Epilepsy.com.stats}. 

A recent review \cite{Allers2015} showed that epilepsy "creates a substantial burden on households" through either loss of productivity or out of pocket costs. While the review noted that the burden could be offset by health insurance, the upfront costs associated with treatment are significant. In Italy, the annual cost per patient is \$1,736, while in Spain it was \$2,813, not accounting for a \$2,924 surgery consultation fee. While these costs represent health system costs in socialized medical care societies, and are therefore shared over a larger population, the case is less forgiving in countries where individual insurance is the norm. In the US, out of pocket cost for a hospital stay is \$1,018. The review also considered indirect costs by associating productivity and employment status changes. While employment status did not differ after the diagnosis, productivity losses accounted for an annual average of \$2,037 in Spain, and \$2,146 per 3 months in Germany. The review concluded that the highest costs of epilepsy treatment come from surgery. A different study \cite{vivas2012health} showed that hospital charges for epilepsy patients has increased by 137
.9\% between 1993 and 2008, rising from \$10,050 to \%23,909 while the average length of stay decreased from 5.9 to 3.9 days.

Epilepsy also affects patients' lives in other ways. Patients become restricted to their homes due to a fear of having a seizure during their child's recital, work, or even grocery shopping. Even if the fear of embarrassment subdues, patients take a risk when performing activities that put them in a position in which they can endanger them and the people around them. For example, deciding to ride a bike becomes a more challenging preposition since the patient could suffer from a seizure and endure serious injuries while biking. More importantly, driving is an increasingly risky activity with seizure prone patients. Driving laws make it difficult for patients to live an ordinary individual life with driving restrictions on patients who suffer from seizures. Each state in the United States has its own laws regarding the requirements that patients must meet in order to be allowed to drive on the road. In most cases the patients physician is involved in the decision on whether or not they should be allowed to drive. 

\subsection{Treatment}

No real cure for the dysfunction of electrical activity in the brain has been developed, therefore epilepsy has no real cure as of yet. Most treatments attempt to mitigate the symptoms (seizure) by using a combination of strategies including the reduction of brain activity through medication, removal dysfunctional parts of the brain, and changing the chemistry of the body through restrictive diets. The following gives a summary of each of these strategies and the ways in which they work.

\subsubsection{Medication}
Medication is the first and most common treatment prescribed by doctors for epilepsy cases. Medications differ for different cases of epilepsy, and there is no \textit{one} specific medication that is commonly prescribed. Such medications, commonly called AED (Anti-Epileptic Drugs) can completely control seizures in 7 our of 10 patients. AEDs work by suppressing seizures that are caused by epilepsy, and do not cure or address the true underlaying causes of epilepsy. Epilepsy medication is known to only be effective if taken regularly. AEDs commonly induce side effects in patients, some serious like mental slowness, hepatotoxicity, dizziness, and drowsiness, and some minor ones like skin rashes and weight gain \cite{AEDsideeffects}. AEDs treatments usually continue until the patient is proven to no longer suffer from epileptic seizures.
% TODO: should probably mention how medication works here.

\subsubsection{Surgery}
Surgery is usually performed on patients who suffer from partial epilepsy and on whom medication has not been affective. However, decisions to have surgery are being made sooner as some correlation has been shown between how early the surgery is performed and its success rate \cite{Epilepsy.com.surgery}. Epilepsy surgeries are split into two main categories: resection and disconnection surgeries. Resection surgeries entails removing the part of the brain that causes electric dysfunction and therefore seizures, sometimes resulting in a complete 'cure' from epilepsy. Such a procedure can have the immediate side effect of losing brain functions or memories like being able to play the Piano or the memory of a child's first word. On the other hand, the goal of disconnection surgeries is to cut nervous pathways that are thought to cause epilepsy in the specific patient. Disconnection surgery is often undertaken if seizure is caused by a vital part of the brain and does not provide a 'cure' from epilepsy but rather a relief for the patient.
Even more than typical surgeries, brain surgery can be highly risky and physicians advice it as a final resort. This risk draws a sharp contrast to the convenience of medication, since most AEDs can be administered easily and require only an adherence to the drug schedule. Therefore, medication is recommended before any surgical approach is considered.

\subsubsection{Diet}
Dietary restrictions such as fasting have been used to mitigate the onset of epilepsy since biblical times \cite{bailey2005use}, and recent developments in the nutrition perspective on epilepsy have resulted in the resurgence of what is now known as the ketogenic diet. This diet consists of high fat and low carbohydrate items and focuses on mimicking the body's reaction to fasting by using fatty acids as a main source of fuel for the body. The diet works remarkably well on infants who have a mutation that affects the transport of Glucose (a main body fuel source): with the ketogenic diet helps prevent microcephaly, mental retardation, spasticity, and ataxia as a consequence of relative brain hypoglycemia (lack of glucose). This bolsters the evidence that the ketogenic diet regulates the body to use fatty acids as an alternative fuel for the brain. The ketogenic diet was also shown to help Alzheimer Disease patients, though that might have to do with eating less carbohydrates than eating more fatty acids \cite{baranano2008ketogenic}.

% COUD POSSIBLE ADD MORE ABOUT ALTERNATIVE MEDICINE (http://www.epilepsy.com/learn/treating-seizures-and-epilepsy/complementary-health-approaches)