\section{Sugihara Causality}
The concepts of abstract correspondence, correlation and interpreting causation has been discussed in philosophical literature at least as early as Berkley's and Locke's arguments on human perception \cite{Locke1841} \cite{Berkeley1874}. Until now, the debate focused on what constitutes a causative effect and how such an effect might be discerned. From philosophy, the debate has moved to empirical science, where different models of causality have been proposed, none of which has been declared the true standard. Causality is often mistaken for correlation because the relationship between the two is not always clear. For example, falling down the stairs could be correlated with breaking a limb, but is falling down the stairs caused by breaking a limb, or is breaking a limb caused by falling down the stairs? One might argue that a person wouldn't break a limb without falling down the stairs, but what if a limb was stressed enough that it broke while a person was normally walking down the stairs, causing the person to lose control of their leg and therefor fall? In this sense, causality is a murky subject that is often considered to be highly subjective and at times nondetachable from the context of the situation. With that being said, there have been many attempts to address the concept of causality in mathematics in the form of the causality of continuous signals on one another. For example, if one could attain the stock prices for hot dogs and hot dogs buns, one would be able to hypothetically see a correlation between the sales of hot dogs and hot dog buns. It is also clear that one can describe a logical causation between the two stock prices, since the products are complementary to one another: if the price for one goes down, the price for the other goes down as well. For the case when substantial fluctuations happen in the stock price for both products, how would one determine which product caused the change in the price of the other? Surely the price drop was caused by \textit{something}, so which of the products caused the other? This is where causality models come into play---this is an example of the use of causality models in the field of Economy.

A particular causality model, Granger Causality (GC), has been widely used in application in the econometric fields \cite{Granger1969}, and has been the de facto model when causality is concerned. The way Granger Causality works is beyond the scope of this paper, but suffice it to say that Granger Causality behaves best in linear, stochastic systems. However, this causality model carries its own limitations, as, even with extensions to non-linear systems, it has generally not been seen capable of inferring causality in deterministic systems where feedback loops and nonlinearity are a defining feature. New models of causality have been introduced to attempt to go beyond these limitations. Dynamic Bayesian Netowrks and, more recently, the Convergent Cross Mapping (CCM) are some such models. 

The CCM model relies convergence of distance of nearest neighbors in the shadow manifold of pairs of variables \cite{Sugihara2012}. A shadow manifold of variable $\omega$ is an $E$ dimensional reconstruction of $E$ delayed signals of $\omega$. Each of these signals is delayed by a scalar multiple of $E\tau$ such that shadow the manifold of $\omega$, $M_{\omega}$ is described as 
$$M_{\omega} = f\Big(\omega(t), \omega(t-\tau), \omega(t-2\tau), \dots, \omega(t-(E-1)\tau)\Big)$$.
Applying Takens' embedding theorem, it can be shown that each shadow manifold of a variable is a projection of the dynamic system's manifold, $M$, that preserves the topology of $M$ \cite{Dixon1999,Deyle2011,Takens1981}. For example, in a dynamic system like the Lorenz Attractor where the dynamics of each variable is affected by the other variables in the system, it can be said that each variable subscribes to the overall dynamic of the system. Therefore, the state of one variable could be used to infer the state of another variable if they are dynamically linked. 

Using this feature of dynamic systems, the CCM model infers causality from the convergence of prediction of one variable's state based on another's as $L$ increases, where $L$ is the length of data points considered in the prediction. This implies that $L$ needs to be sufficiently large to allow an observation of convergence. This convergence is the test used to determine Sugihara Causality, named after its author who describes it as a required but not complete definition of causality \cite{Sugihara2012}. This approach is the first step towards more general and applicable causality models since GC. Since the introduction of CCM, it has been shown to be successfully predictive in biological \cite{Deyle16042013,Wang2014,Sugihara2012,Mcbride2015,Nes2015} and cosmological \cite{Tsonis2015} applications while showing weaknesses in others \cite{Mccracken2014}. 

Extensions to and amalgamations of the CCM model are beginning to surface in literature. Clark \textit{et al.} proposed an extension to CCM that relies on measuring the smoothness of the mapping (also called flow) function $\phi$, thereby reducing the $L$ length requirement\cite{Clark2015}.  Wismüller \textit{et al.} proposed a Mutual Connectivity Analysis framework for the "analysis and visualization of non-linear functionalconnectivity in the human brain from resting state functional MRI" \cite{wismuller2014} which relies heavily on CCM. Tajima \textit{\textit{et al.}} use the fundamental idea of state space reconstruction to find two measures. The first is \textit{Complexity} which is the best embedding dimension for a certain signal (embedding dimension at which the cross mapping is saturated). The second is \textit{directionality}, the difference in cross map skill or embeddedness between two a pair of signals. With those two measures, they show that the brain exhibits different complexities during conscious and unconscious states. Here, we explore the application of CCM in estimating the causality between neuronal regions by constructing a network of pairwise causality. We then analyser features of such networks during normal and epileptic seizure periods. We attempt to localize the origin of seizures as well as predict their occurance by using the properties of causal networks.

