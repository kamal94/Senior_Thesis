\section{Introduction}
Machine learning is a field that stretches across computer science, mathematics, and even psychology. One of its main purposes is to teach computers how to perform actions without explicitly programming the computers how to perform the actions correctly. For example, instead of programming a computer to play checkers by accounting for every possible situation and move---implementing them with if/else statements---a machine learning approach would be to let the computer play many games and attempt somewhat random moves, keeping track of which moves were beneficial and which were malevolent. Using this approach, the programmer does not need to account for the incomprehensible possibilities for all checkers boards, but rather needs to implement a learning algorithm that would be able to learn from the moves it performs. This is where computer science intersects psychology, since most if not all of the machine learning approaches are inspired by how humans perceive a problem and approach it. These approaches to describe, replicate, and augment the human approach to solving problems are manifested in mathematical frameworks, which is where computer science intersects mathematics in this field. 

Since its beginning in the early 19th century, the field of machine learning has been gradually applied to solving many problems in computer science. Some of these problems include optical character recognition, spam detection, weather prediction, fraud detection, and language understanding \cite{RobSchapire}. Recently, some breakthroughs in neural networks, a specific branch of machine learning, has resulted in an increased interest in machine learning and its real-life applications. This recent development---called deep learning---is being applied to develop autonomous driving vehicles, read people's lips \cite{assael2016lipnet}, Compress images more efficiently \cite{Lucas20016}, simulate gas and liquid physical spread \cite{tompson2016accelerating}, and literally judge books by their covers \cite{Iwana2016}. Such projects might be advanced, but the machine learning community is bridging the gap between the highly academic research oriented projects, high profit industry oriented endeavors, and the computer science labor force that is severely lacking in machine learning expertise. Through websites like \textit{Kaggle.com}, \textit{r2d3.com}, \textit{deeplearning.net}, \textit{deeplearningbook.org}, and \textit{neuralnetworksanddeeplearning.com}, computer scientists can learn the different ways that machine learning can be applied, and have access to resources and competitions that gives them experience in the field. During my own journey in the field of machine learning, I came across several problems and ways to approach them with the tool box I developed.

This journey led me to the University of Colorado Colorado Springs (UCCS), where I participated in a Research Experience for Undergraduates (REU) under the supervision of Dr. Rory Lewis and Dr. Peter Erdi. This paper is a culmination of the learning and discovery made during and since the time I spent at UCCS in the summer of 2016. As a requirement of the REU, Machine Learning techniques were to be applied to a problem of my choosing, and after much deliberation with my advisors, I decided to explore the topic of Epilepsy and attempt to apply machine learning, in whatever extent possible with my knowledge, to address problems in epilepsy treatment. Through this paper, I guide the reader through the topics of Epilepsy, Machine Learning, and Sugihara Causality, amalgamating them into one coherent piece that is used to describe brain states. This paper assumes a certain level of literacy in readership, mainly that of medium level Computer Science and Mathematics background. Whenever possible, complicated terms and ideas will be distilled to simpler ones that may be more familiar to the reader. The paper will begin by describing epilepsy and putting it into the context of our everyday lives. Then, the topic of machine learning will be introduced, giving the reader a broad (but non-exhausting) overview of some popular techniques used. This section will be the most intensive one in this paper, due to two main reasons: since the majority of the time I spent in UCCS has been in discovering the fascinating land of machine learning, I felt it was only appropriate that this paper fully reflects the majesty I found in my search; since machine learning is a mathematically involved subject that most computer scientists might not be familiar with, I believe it is appropriate to thin the content out for the reader to ensure a proper digestion of the information provided. After Machine Learning, the last of the trio of subjects, Sugihara Causality, will be introduced, shinning light into a subject that is not commonly discussed in Machine Learning, Mathematics, or Computer Science, but nevertheless an important piece of my work in UCCS. After the introduction into these three areas of research, this paper will then walk the reader through my attempt at putting the three areas together to make sense of how areas of the brain communicate with one another, and thereby help diagnose and track epilepsy in the brain.



% Machine Learning is the science of training a machine to recognize certain patterns in the given data. There are 3 main subcategories of ML, namely supervised learning, unsupervised learning, and reinforcement learning. While all of these categories have been used in the biological field, my own work has been done in supervised and unsupervised learning. The data of my work is comprised of voltage signals collected through electroencephalogram (EEG) electrodes inserted into specific regions of the brain. As the brain alters in state, the voltage fluctuates, and these fluctuations are recorded by this electrode. Due to the nature of epilepsy, only one electrode is typically required to identify the stages of an epileptic seizure. However, our thesis is that recording of multiple EEG electrodes in the brain could help not only identify epilepsy but also localize its origin in the brain.

